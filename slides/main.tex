% ===============================
%   Higgs Mechanism Seminar
%   main.tex  (in slides/)
% ===============================

\documentclass{beamer}

% Preamble with theme, packages, macros
% ===============================
%        PREAMBLE FOR BEAMER
% ===============================

% -------------------------------
%         DOCUMENT SETTINGS
% -------------------------------
\usepackage[T1]{fontenc}
\usepackage[utf8]{inputenc}
\usepackage{lmodern}               % Clean font
\usepackage{microtype}             % Better typography
\usefonttheme{professionalfonts}
\usepackage[english]{babel}
\usepackage{csquotes}
\usepackage{ragged2e}

% -------------------------------
%             BEAMER
% -------------------------------
\mode<presentation>{
  \usetheme{default}               % clean base; editable later
  \usecolortheme{dove}             % minimalistic greyscale
  \setbeamertemplate{navigation symbols}{}      % remove nav buttons
  \setbeamertemplate{footline}[frame number]    % tiny frame # only
  \setbeamertemplate{caption}[numbered]
}

% Make colors subtle and modern
\definecolor{myblue}{HTML}{1B63A6}
\definecolor{myred}{HTML}{C0392B}
\definecolor{mygreen}{HTML}{27AE60}

\setbeamercolor{normal text}{fg=black!92,bg=white}
\setbeamercolor{structure}{fg=myblue}
\setbeamercolor{title}{fg=myblue}
\setbeamercolor{frametitle}{fg=myblue}
\setbeamercolor{subtitle}{fg=gray!65}
\setbeamercolor{background canvas}{bg=white}

% -------------------------------
%              MATH
% -------------------------------
\usepackage{amsmath, amsfonts, amssymb, bm}
\usepackage{mathtools}
\usepackage{physics}               % bra-ket, \dv, \pdv, \comm etc.
\usepackage{tensor}                % index notation
\usepackage{siunitx}               % units
\sisetup{detect-all}

\AtBeginDocument{\RenewCommandCopy\qty\SI}

% Useful math macros
\newcommand{\lag}{\mathcal{L}}
\providecommand{\vev}[1]{\langle #1 \rangle}
\newcommand{\SU}{\mathrm{SU}}
\newcommand{\U}{\mathrm{U}}

% -------------------------------
%          TIKZ + GRAPHICS
% -------------------------------
\usepackage{tikz}
\usetikzlibrary{arrows.meta, calc, patterns, decorations.markings}
\usepackage{pgfplots}
\pgfplotsset{compat=1.18}

\usepackage{graphicx}
\graphicspath{{figures/}}         % images live here

% Improved figure alignment
\setlength{\tabcolsep}{4pt}

% -------------------------------
%          BIBLIOGRAPHY
% -------------------------------
\usepackage{biblatex}
\addbibresource{../bibliography.bib}

% -------------------------------
%        ITEMIZE & SPACING
% -------------------------------
\linespread{1.05}
\setlength{\parskip}{5pt plus 1pt minus 1pt}
\setlength{\parindent}{0pt}

\setbeamertemplate{itemize item}{\color{myblue}\textbullet}
\setbeamertemplate{itemize subitem}{\color{myblue}–}

% -------------------------------
%      TITLE & SECTION STYLE
% -------------------------------
\setbeamerfont{frametitle}{size=\LARGE,series=\bfseries}
\setbeamerfont{title}{size=\Huge,series=\bfseries}
\setbeamerfont{subtitle}{size=\normalsize,series=\mdseries}
\setbeamerfont{framesubtitle}{size=\normalsize,series=\mdseries}
\setbeamersize{text margin left=16mm,text margin right=16mm}

% Center frametitles
\setbeamertemplate{frametitle}{
  \nointerlineskip
  \begin{beamercolorbox}[wd=\paperwidth,center]{frametitle}
    \vspace{3mm plus 1mm minus 0.5mm}
    \parbox{0.9\paperwidth}{\centering\usebeamerfont{frametitle}\color{myblue}\insertframetitle\par}
    \vspace{0.75mm}
  \end{beamercolorbox}
}

% -------------------------------
%   FULLSCREEN IMAGES (key for minimal words)
% -------------------------------
\newcommand{\fullimage}[1]{
  {
    \usebackgroundtemplate{\includegraphics[width=\paperwidth]{#1}}
    \begin{frame}[plain]{}
    \end{frame}
    \usebackgroundtemplate{} % reset
  }
}

% -------------------------------
%   CLEAN SLIDE (no frame title)
% -------------------------------
\newenvironment{cleanframe}{
  \setbeamertemplate{frametitle}{}
  \begin{frame}
}{
  \end{frame}
  \setbeamertemplate{frametitle}{
    \nointerlineskip
    \begin{beamercolorbox}[wd=\paperwidth,center]{frametitle}
      \vspace{3mm plus 1mm minus 0.5mm}
      \parbox{0.9\paperwidth}{\centering\usebeamerfont{frametitle}\color{myblue}\insertframetitle\par}
      \vspace{0.75mm}
    \end{beamercolorbox}
  }
}

% -------------------------------
%  TITLE PAGE LOOK
% -------------------------------
\defbeamertemplate*{title page}{custom}{
  \begin{centering}
    \vspace{1.6cm}
    {\color{myblue}\rule{0.72\paperwidth}{0.5pt} \par}
    \vspace{0.55cm}
    {\color{myblue}{\Huge\bfseries \inserttitle} \par}
    \vspace{0.3cm}
    {\large\color{gray!65} \insertsubtitle \par}
    \vspace{0.6cm}
    {\normalsize \insertauthor \par}
    {\small \insertinstitute \par}
    \vspace{0.35cm}
    {\scriptsize \insertdate \par}
    \vspace{0.55cm}
    {\color{myblue}\rule{0.72\paperwidth}{0.5pt} \par}
  \end{centering}
}
\setbeamertemplate{title page}[custom]



% --------------------------------
%        METADATA
% --------------------------------
\title[Higgs Mechanism]{The Higgs Mechanism and the Origin of Mass}
\subtitle{2013 Nobel Prize in Physics}
\author{Vahid Gorgin}
\institute{[Your University / Department]}
\date{\today}

% --------------------------------
%        DOCUMENT
% --------------------------------

%%%%%%%%%%%%%%%%%%%%%%%%%%%%%%%%%%%%%%%%%%%%%%%%
% Higgs seminar slide layout macros
%%%%%%%%%%%%%%%%%%%%%%%%%%%%%%%%%%%%%%%%%%%%%%%%

% Vertical gap macro for consistent rhythm
\newcommand{\slidegap}{\vspace{0.7em}}

% Single centered phrase, tuned for minimal-text slides
\newcommand{\slidephrase}[1]{%
  \par\slidegap
  \begin{center}\Large #1\end{center}%
  \slidegap\par
}

% Smaller sub-phrase (e.g. second or third line)
\newcommand{\slidesubphrase}[1]{%
  \par\slidegap
  \begin{center}\normalsize #1\end{center}%
  \par
}

% Equation block with symmetric spacing
\newcommand{\slideeq}[1]{%
  \par\slidegap
  \[
    #1
  \]
  \slidegap\par
}

% Standardized figure block (e.g. discovery plots)
\newcommand{\slidefigure}[2][0.8]{%
  \par\slidegap
  \begin{center}
    \includegraphics[width=#1\linewidth]{#2}
  \end{center}
  \par
}

% Two-line “tension” structure: statement + consequence
\newcommand{\slidetension}[2]{%
  \par\slidegap
  \begin{center}
    \Large #1\\[0.9em]
    \normalsize #2
  \end{center}
  \slidegap\par
}

%%%%%%%%%%%%%%%%%%%%%%%%%%%%%%%%%%%%%%%%%%%%%%%%
\begin{document}

% ---------- Title Page ----------
\begin{frame}
  \titlepage
\end{frame}

% ---------- Optional Outline (can remove later) ----------
% \begin{frame}{Roadmap}
%   \tableofcontents
% \end{frame}

% ---------- Content ----------
% content/titlepage.tex

\begin{cleanframe}
  \slidephrase{Why do particles have mass?}
  \slidesubphrase{And what if our best theory were missing the very field that gives it?}
\end{cleanframe}

\begin{frame}{A strange universe}
  \slidephrase{Imagine a world where all particles move at the speed of light.}
  \slidesubphrase{Mass slows everything down; something hidden must be doing the work.}
\end{frame}

\begin{frame}{The missing piece}
  \slidetension{For decades, the Higgs boson was a rumor.}{Could an invisible field be real, or would the Standard Model unravel?}
  \slidesubphrase{Tonight: follow the trail from paradox to discovery.}
\end{frame}
              % cold open / hook, if you separate it
% content/part1_problem.tex

\begin{frame}{Gauge symmetry vs.\ mass}
  \centering
  \Large
  Gauge symmetry forbids mass terms.
  
  \vspace{1em}
  \[
    \lag_{\text{gauge}} = -\frac{1}{4} W_{\mu\nu} W^{\mu\nu}
    \quad\Rightarrow\quad m_W = 0
  \]
\end{frame}

\begin{frame}{But experiments disagree}
  \centering
  \Large
  We measure massive $W$ and $Z$ bosons.
  
  \vspace{1em}
  \Large
  Theory says: \emph{no mass}. \\
  Experiments say: \emph{big mass}.
\end{frame}

\begin{cleanframe}
  \centering
  \Huge
  Adding mass by hand breaks the theory.
\end{cleanframe}

\begin{frame}{The crisis}
  \centering
  \Large
  Naive mass terms destroy gauge symmetry, \\
  renormalizability, and predictivity.

  \vspace{1em}
  \Large
  How can a gauge theory have massive bosons?
\end{frame}
          % "Why do particles have mass?" + conflict
\begin{frame}{A symmetry… but a broken ground state}
  \centering
  Symmetric equations.
  Ground state refuses the symmetry.
  Conflict begins.
\end{frame}

\begin{frame}{A Simple Scalar Field}
  \centering
  Complex scalar field with symmetric dynamics.
  \[
    \mathcal{L} = \partial_\mu \phi^* \partial^\mu \phi - \mu^2 |\phi|^2 - \lambda |\phi|^4
  \]
\end{frame}

\begin{frame}{The Potential}
  \centering
  \[
    V(\phi) = \mu^2 |\phi|^2 + \lambda |\phi|^4
  \]
  If $\mu^2 < 0$, the minimum lies away from $\phi = 0$.
\end{frame}

\begin{frame}{Choosing a Vacuum}
  \centering
  The vacuum picks a direction in field space.
  The symmetry is hidden, not destroyed.
\end{frame}

\begin{frame}{Goldstone Modes}
  \centering
  Breaking a continuous symmetry creates massless excitations along the flat directions.
  Goldstone theorem at work.
\end{frame}

\begin{frame}{Toward Massive Gauge Bosons}
  \centering
  Gauge fields can absorb Goldstone modes.
  Hidden symmetry gives mass to vector bosons.
\end{frame}

\begin{frame}{Enter the Higgs Field}
  \centering
  Next: build the electroweak story where the Higgs field completes mass generation.
\end{frame}
 % gauge symmetry vs mass; crisis
\begin{frame}{A field that fills space}
  \slidephrase{Introduce the Higgs as an $SU(2)$ doublet.}
  \slidesubphrase{A field that is nonzero everywhere in the vacuum.}
  \slidesubphrase{Its presence reshapes the behavior of gauge and matter fields.}
\end{frame}

\begin{frame}{Choosing the Vacuum}
  \slidephrase{We choose a vacuum $\langle \phi \rangle = (0, v/\sqrt{2})^{\mathrm T}$.}
  \slidesubphrase{One direction in the doublet acquires a constant value.}
  \slidesubphrase{Here we visualize this with the Higgs potential.}
\end{frame}

\begin{frame}{Gauge Boson Masses}
\centering
Gauge fields absorb Goldstone modes and become massive.
\[
 m_W = \tfrac{g v}{2}, \qquad m_Z = \tfrac{v}{2}\sqrt{g^2 + g'^2}
\]
Mass arises from interacting with the constant Higgs background.
\end{frame}

\begin{frame}{Fermion Masses}
\centering
Yukawa term: $-y_f\, \bar{\psi}_L \Phi \psi_R + \text{h.c.}$
VEV turns Yukawa coupling into mass.
\[
 m_f = \dfrac{y_f v}{\sqrt{2}}
\]
\end{frame}

\begin{frame}{One Scalar Remains}
  \slidephrase{Goldstone modes are eaten by the gauge fields.}
  \slidesubphrase{One physical scalar excitation is left.}
  \slidesubphrase{This is the Higgs boson we aim to discover.}
\end{frame}

\begin{frame}{Why This Matters Experimentally}
  \slidephrase{Masses and couplings are now predicted.}
  \slidesubphrase{Production and decay rates follow from the Higgs mechanism.}
  \slidesubphrase{These patterns define concrete search channels.}
\end{frame}

\begin{frame}{Can we see the Higgs?}
  \slidephrase{A scalar with predicted couplings.}
  \slidesubphrase{Specific rates into $ZZ$, $\gamma\gamma$, and other channels.}
  \slidesubphrase{ATLAS and CMS can test this picture.}
\end{frame}
       % Higgs field, potential, SSB
% content/part4_discovery.tex

\begin{frame}{From idea to experiment}
  \centering
  \Large
  1964: mechanism proposed. \\
  1980s--2000s: colliders push the energy frontier. \\
  LHC built to finally test the Higgs idea.
\end{frame}

\begin{frame}{The search at the LHC}
  \centering
  \Large
  Look for a new scalar around $\sim 100$--$\SI{1000}{\giga\electronvolt}$.

  \vspace{0.5em}
  \Large
  Decay channels: $\gamma\gamma$, $ZZ^*$, $WW^*$, $\tau\tau$, $b\bar b$.
\end{frame}

\begin{frame}{Discovery: July 4, 2012}
  \centering
  \includegraphics[width=0.8\linewidth]{figures/atlas_hgg_discovery_massplot.png}
  
  \vspace{0.5em}
  \Large
  New boson at $\sim \SI{125}{\giga\electronvolt}$ \\
  with properties consistent with a Higgs.
\end{frame}



\begin{cleanframe}
  \centering
  \Huge
  The last missing piece of the Standard Model.
\end{cleanframe}
        % LHC, 125 GeV discovery
\begin{frame}{The Higgs couples in proportion to mass}
  \centering
  $g_{Hff}\propto m_f,\; g_{HVV}\propto m_V^2$ \\
  Interaction strength rises with particle mass.
\end{frame}

\begin{frame}{Production}
  \centering
  ggF: gluon loop produces $H$, dominant at 13 TeV. \\
  VBF: $qq\to qqH$ via $W/Z$ exchange; forward jets tag the event. \\
  Associated: $VH$ or $ttH$ radiate a Higgs alongside gauge bosons or top pairs.
\end{frame}

\begin{frame}{Decays}
  \centering
  Branching ratios follow couplings and phase space: $\Gamma\sim g^2$ with thresholds shaping the spectrum.
\end{frame}

\begin{frame}{Why $H\to\gamma\gamma$?}
  \centering
  Loop-induced through $W$ and top; rare yet calculable. \\
  Two isolated photons give a crisp electromagnetic peak.
\end{frame}

\begin{frame}{Why $H\to ZZ\to4\ell$?}
  \centering
  Four charged leptons fully reconstruct the decay; backgrounds are tiny. \\
  The invariant mass shows a sharp, resolvable resonance.
\end{frame}

\begin{frame}{The 125 GeV Fingerprint}
  \centering
  Observed rates in $\gamma\gamma$ and $ZZ\!\to\!4\ell$ match predictions.\\[0.7em]
  Coupling strengths follow mass–proportional patterns.\\[0.7em]
  Phenomenology and discovery align.
\end{frame}

\begin{frame}{A consistent picture emerges}
  \centering
  Production and decay patterns fit a $125$ GeV Higgs boson.\\[0.7em]
  Theory, phenomenology, and data form a unified story.\\[0.7em]
  This concludes the arc from mechanism to discovery.
\end{frame}
    % extended Higgs, open questions
% content/ending.tex

\begin{cleanframe}
  \centering
  \Huge
  The Higgs solved one mystery \\
  and opened many more.
\end{cleanframe}

\begin{frame}{Takeaway}
  \centering
  \Large
  Mass is not an input; \\
  it is a consequence of the vacuum structure.

  \vspace{0.8em}
  \Large
  The 2013 Nobel Prize celebrates this shift.
\end{frame}

\begin{cleanframe}
  \centering
  \Huge
  Thank you. \\
  \vspace{0.8em}
  \Large
  Questions?
\end{cleanframe}
                 % final punchline + thank you

% ---------- Bibliography (if you want a refs slide) ----------
\begin{frame}[allowframebreaks]{References}
  \printbibliography
\end{frame}

\end{document}
