\begin{frame}{Why is mass a problem?}
  \centering
  Gauge theories prefer massless fields.
  Yet nature gives $W$, $Z$, and fermions mass.
  Where does mass fit without breaking symmetry?
\end{frame}

\begin{frame}{Gauge bosons and symmetry}
  \centering
  Yang--Mills kinetic term keeps the gauge symmetry.
  \[
    \mathcal{L}_{\text{YM}} = -\frac{1}{4} F_{\mu\nu}^a F^{a\,\mu\nu}
  \]
  Symmetry protects masslessness.
\end{frame}

\begin{frame}{The forbidden mass term}
  \centering
  \[
    \mathcal{L}_{\text{Proca}} = \frac{1}{2} m^2 A_\mu A^\mu
  \]
  This term breaks gauge invariance.
\end{frame}

\begin{frame}{Fermion mass trouble}
  \centering
  \[
    \mathcal{L}_{\text{Dirac}} = -m \overline{\psi}\, \psi
  \]
  Left and right chiralities transform differently under $SU(2)\times U(1)$.
\end{frame}

\begin{frame}{The conflict}
  \centering
  Gauge symmetry forbids naive mass.
  Experiments require massive $W$, $Z$, and fermions.
  A new idea is needed.
\end{frame}

\begin{frame}{A clue: symmetry can hide itself}
  \centering
  Equations stay symmetric.\\[0.7em]
  Ground state can be asymmetric.\\[0.7em]
  Next: a simple scalar field that hides symmetry.
\end{frame}
