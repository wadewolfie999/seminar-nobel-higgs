\begin{frame}{Why is mass a problem?}
  \slidephrase{Gauge theories love massless fields.}
  \slidesubphrase{But nature is full of massive particles.}
  \slidesubphrase{Naive mass terms seem to break our symmetries.}
\end{frame}

\begin{frame}{Gauge bosons and symmetry}
  \centering
  Yang--Mills kinetic term keeps the gauge symmetry.
  \[
    \mathcal{L}_{\text{YM}} = -\frac{1}{4} F_{\mu\nu}^a F^{a\,\mu\nu}
  \]
  Symmetry protects masslessness.
\end{frame}

\begin{frame}{The forbidden mass term}
  \centering
  \[
    \mathcal{L}_{\text{Proca}} = \frac{1}{2} m^2 A_\mu A^\mu
  \]
  This term breaks gauge invariance.
\end{frame}

\begin{frame}{Fermion mass trouble}
  \centering
  \[
    \mathcal{L}_{\text{Dirac}} = -m \overline{\psi}\, \psi
  \]
  Left and right chiralities transform differently under $SU(2)\times U(1)$.
\end{frame}

\begin{frame}{The conflict}
  \slidetension{Gauge symmetry forbids naive mass.}{Experiments demand massive $W$, $Z$, and fermions.}
  \slidesubphrase{We need a new idea that keeps the symmetry but generates mass.}
\end{frame}

\begin{frame}{A clue: symmetry can hide itself}
  \slidephrase{Equations stay symmetric.}
  \slidesubphrase{The ground state can be asymmetric.}
  \slidesubphrase{Next: a simple scalar field that hides symmetry.}
\end{frame}
