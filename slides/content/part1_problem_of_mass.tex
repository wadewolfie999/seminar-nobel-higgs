% content/part1_problem.tex

\begin{frame}{Gauge symmetry vs.\ mass}
  \centering
  \Large
  Gauge symmetry forbids mass terms.
  
  \vspace{1em}
  \[
    \lag_{\text{gauge}} = -\frac{1}{4} W_{\mu\nu} W^{\mu\nu}
    \quad\Rightarrow\quad m_W = 0
  \]
\end{frame}

\begin{frame}{But experiments disagree}
  \centering
  \Large
  We measure massive $W$ and $Z$ bosons.
  
  \vspace{1em}
  \Large
  Theory says: \emph{no mass}. \\
  Experiments say: \emph{big mass}.
\end{frame}

\begin{cleanframe}
  \centering
  \Huge
  Adding mass by hand breaks the theory.
\end{cleanframe}

\begin{frame}{The crisis}
  \centering
  \Large
  Naive mass terms destroy gauge symmetry, \\
  renormalizability, and predictivity.

  \vspace{1em}
  \Large
  How can a gauge theory have massive bosons?
\end{frame}
